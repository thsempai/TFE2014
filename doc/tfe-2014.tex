\documentclass[a4paper,12pt]{report}

\usepackage{tfe-style}


%meta-donnée
\author{Thomas Stassin}
\date{}

\begin{document}

% première page de titre
\begin{titlepage}
\vfill

\parpic[l]{\includegraphics[scale=1.0]{ifosup-logo.png}}

\begin{center}
\Large{Institut de Formation Supérieure de la Ville de Wavre}\\

\textbf{\Huge{IFOSUP}}
\end{center}

\begin{center}
\LARGE{Bachelier en informatique de gestion.}
\end{center}

\vfill

\begin{center}
\bf
\LARGE{Ankh Fire}\\
\Large{Gestionnaire de tags audios}
\end{center}

\vfill


\begin{flushleft}
Directeur de travail :\\
\textbf{Ph. Robillard}
\end{flushleft}

\begin{flushright}
\bf
Travail de fin d'études présenté par Thomas STASSIN en vue de l'obtention du diplôme de Bachelier en Informatique de Gestion
\end{flushright}

\begin{center}
\textbf{Année académique 2013-2014}
\end{center}


\end{titlepage}

% insertion d'une page blanche
\newpage
\thispagestyle{empty}
\addtocounter{page}{-1}
\null
\newpage

% deuxième page de titre
\begin{titlepage}
\vfill

\parpic[l]{\includegraphics[scale=1.0]{ifosup-logo.png}}

\begin{center}
\Large{Institut de Formation Supérieure de la Ville de Wavre}\\

\textbf{\Huge{IFOSUP}}
\end{center}

\begin{center}
\LARGE{Bachelier en informatique de gestion.}
\end{center}

\vfill

\begin{center}
\bf
\LARGE{Ankh Fire}\\
\Large{Gestionnaire de tags audios}
\end{center}

\vfill


\begin{flushleft}
Directeur de travail :\\
\textbf{Ph. Robillard}
\end{flushleft}

\begin{flushright}
\bf
Travail de fin d'études présenté par Thomas STASSIN en vue de l'obtention du diplôme de Bachelier en Informatique de Gestion
\end{flushright}

\begin{center}
\textbf{Année académique 2013-2014}
\end{center}


\end{titlepage}

\tableofcontents
\pagebreak

\chapter{Introduction}

Dans le cadre de mes cours d’informatique de gestion à l’IFOSUP, je dois réaliser un travail de fin d’études. Je donc choisis de faire une application de gestion de fichiers audios afin de pouvoir mettre en pratique les connaisances acquises durant ses trois années d'études.

C'est pour moi aussi l'occasion de réaliser un projet du début à la fin, étant à la fois analyste fonctionnel, analyste technique et devellopeur.\\

De plus, la conception de cette application va me permettre de travailler sur avec des technologie qui me tiennent à c\oe{}ur et de les éprouver dans une situation "professionnelle".

Les multiples facettes de ce programme (un coté client et autre server) vont me confronter avec plusieurs aspect du devellopement informatique dans le but qu'à la fin de la réalisation de ce travail, j'en resorte avec le maximun d'expérience concrète.


\chapter{Contexte}

Le développement d'une application de gestion de fichiers audios, permettant leur lecture et leur classification à l'aide de tags \footnote{Mot anglais désigant une "étiquette" (ou un mot-clé) associé ou assigné à des données (dans ces cas ci un fichier audio.}.

Cette application servira à mon client, pour créer des listes de lectures rapidement, afin de pouvoir diffuser des musiques adéquates lors d'événements qui requièrent plusieurs styles d'ambiances audios.

Cette application sera développée pour pouvoir être deployée sur plusieurs plateforme (Linux, Mac-OS, Windows) et pourra se connecter à un serveur pour pouvoir partager et récupérer les tags lier au fichiers audio.

\section{Le client}

Mon client est un particulier, qui organise des événements où la musique d'ambiance est un élément important du "décor". \\

Un exemple de ce type d'événement est une soirée jeu de rôles.
Lors de ce genre de soirée, l'ambiance sonore est souvent appréciée et si elle est présente, sa gestion doit être fluide, afin de ne pas créer de rupture dans l'ambiance.

\section{La demande initiale}
Mon client cherchait une application qui permettait de taguer des musique et créer automatiquement des listes de lectures basées sur les tags.\\

Cette application devrait aussi proposer de pouvoir créer un "tableau de bord" qui permettrais de passer d'une liste de lecture à une autre (avec un phase out / phase in), cela de manière ergonomique en demandant le moins de manipulation possible.\\

Et enfin, les utilisateurs de cette application pourraient partager leurs tags entre eux, via un serveur qui stockerait les tags enregistrer par chacun et permettrais la diffusion de ceux-ci.

L'application demandée tournera sur les trois Operating System suivant:\\

\begin{itemize}
\item Linux
\item Windows
\item Mac-OS
\end{itemize}

\section{Les produits existants}

En matière de diffusion de musique audios, il y a pléthore de programme existant.
Voici deux exemples que j'ai choisis,car ils ont une caractéristique commune avec l'application demandée: ils sont exécutables sur les 3 OS cité plus haut. \\

\begin{itemize}
\item Quodlibet
\item Amarok
\item Banshee (en Beta sous Mac-OS et en alpha sous Windows)
\end{itemize}

Des logiciels qui permettent de taguer efficacement un fichier audio, en permettant de créer ses propres tags, je n'en ai trouvé qu'un:\\

Ex Falso.
C'est un éditeur de tags, livrer avec Quodlibet.\\

Dans la totalité de ces programmes, aucun ne permet de faire de la création de playlist dynamique, du moins pas de manière poussée, et aucun ne permet la création de "tableau de bord" comme souhaiter par mon client.
De plus, la partie serveur, avec partage des tags serait quelque chose de totalement neuf, n'ayant pas d'équivalence.

\chapter{Cahier des charges}

\section{Elaboration}

% parler du tags ID3 en note de bas de pages ici

\section{Lots d'information identifiés}

\subsection{Lots d'information existants à conserver}

Les fichiers audio existant chez le client seront conservés. De la même façon les tags ID3 déjà présent sur ces fichiers resteront tel quel. 

\subsection{Lots d'information existants à remplacer}

%peut-être détailler un peu plus pourquoi.
Aucun lots d'information existants n'est à remplacer.

\subsection{Lots d'information à produire}
 



\end{document}